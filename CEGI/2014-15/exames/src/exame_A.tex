% !TeX spellcheck = pt

%% ************************************************
%% Universidade Nova de Lisboa
%% NOVA Information Management School
%% Computação em Estatística e Gestão de Informação
%% Júlio Caineta, 2015
%% ************************************************
\documentclass[addpoints]{exam}
\usepackage[utf8]{inputenc}
\usepackage{amsmath}
\usepackage[portuguese]{babel}
\usepackage[hidelinks,pdfusetitle]{hyperref}
\author{Júlio Caineta}
\title{CEGI 2014/2015 - Exame 1ª Época - Versão A}
\usepackage{lastpage}
\usepackage{minted}
\usepackage{gensymb}
\usepackage{wrapfig}
\usepackage{tabulary}
\usepackage{graphicx}
\usepackage{xfrac}
\usepackage{multirow}
\usepackage[phantomtext]{dashundergaps}
%\usepackage{color}
\cfoot{Página \thepage\ de \pageref{LastPage}}
\renewcommand{\solutiontitle}{\noindent\textbf{Solução:}\par\noindent}
\pointpoints{valor}{valores}
\pointsinrightmargin
\bracketedpoints
\marginpointname{ val.}
\renewcommand*\half{.5}
\hqword{Pergunta}
\hpgword{Página}
\hpword{Cotação}
\hsword{Cotação obtida}
\htword{Total}
\newminted{r}{autogobble}
\newmintinline{r}{}
 
\printanswers
%\shadedsolutions

%\titlegraphic{}

\begin{document}
	
\noindent\begin{minipage}{0.2\textwidth}% adapt widths of minipages to your needs
	\includegraphics[width=2cm]{logo.png}
\end{minipage}
\hfill
\begin{minipage}{0.8\textwidth}
	\begin{center}
		\textsc {\small NOVA IMS -- Universidade Nova de Lisboa} \\
		\textsc {Computação em Estatística e Gestão de Informação -- 2º Semestre 2014/15}
	\end{center}
\end{minipage}


\begin{center}
%	\vspace{5mm} \\
	{\large Exame 1ª Época -- Versão A -- 11/04/2015}
\end{center}
 
\vspace{5mm}
\makebox[\textwidth]{Número: \underline{\phantom{XXXXXXXX}} Nome:\enspace\hrulefill}
\vspace{5mm}

{{\large\textbf{Leia, por favor, com atenção:}}
\begin{enumerate}
	\item Este exame deverá ser realizado com acesso a um computador com o programa R instalado. Só poderá aceder ao programa R e ao sistema de ajuda do mesmo.
	\item É proibido o uso de qualquer material de apoio (livros, apontamentos, telemóvel), assim como a troca de qualquer informação com os colegas.
	\item A entrega do exame e a saída da sala só é possível no final do exame .
	\item Deverá escrever o seu nome, número e curso no cabeçalho desta folha
	\item As respostas às questões deverão ser dadas, exclusivamente, na folha do enunciado, no espaço reservado para tal. Estas respostas deverão ser apenas código em \textbf{R}.
	\item O não cumprimento de alguma das regras conduzirá à anulação do exame.
	\item A duração do exame é de 75 minutos.
\end{enumerate}

\vspace{10mm}

\begin{questions}
	\question O vector \rinline{islands} contém as áreas, em milhares de milhas quadradas (1000 mi$^2$), das maiores massas de terra (continente e ilhas). Os sete valores mais altos, referem-se a continentes.
	
	\begin{parts}
		\part[1] Descubra qual é a maior ilha do planeta e a respectiva área.~%\makeemptybox{1in}
		
			\begin{solution}
				\begin{rcode}
					> sort(islands, decreasing=TRUE)[7 + 1]
					Greenland 
					840 
				\end{rcode}
			\end{solution}
			
		\part[1] Sabendo que 1 mi = 1.609344 km, defina uma função que recebe um vector com valores de áreas em 1000 mi$^2$, e devolve um vector com valores de área em km$^2$. Converta os valores do vector  \rinline{islands} para km$^2$.~%\makeemptybox{1in}
		
		\begin{solution}
			\begin{rcode}
				mi2km = function(areas) {
				  areas * (1.609344 ^ 2) * 1000
				}
				mi2km(islands)
			\end{rcode}
		\end{solution}
		
		\part[2] Utilize a função \rinline{substr} para modificar os nomes dos elementos do vector \rinline{islands}, mantendo apenas os primeiros três caracteres de cada nome.~%\makeemptybox{1in}
		
		\begin{solution}
			\begin{rcode}
				names(islands) = substr(names(islands),1,3)
			\end{rcode}
		\end{solution}
			
	\end{parts}
	
	\question[3] O \rinline{data.frame rock}  contém valores de quatro propriedades de amostras de rocha. Encontre as amostras em que os valores de três das suas propriedades são superiores aos valores do respectivo 3º quartil.~%\makeemptybox{1in}
	
	\begin{solution}
		\begin{rcode}
			qs = apply(rock, 2, quantile, .75)
			> rock[apply(rock, 1, function(x) sum(x > qs) == 3),]
			area    peri    shape  perm
			18 10743 4787.62 0.262727  58.6
			22 11876 4353.14 0.291029 142.0
		\end{rcode}
	\end{solution}
	
	\question A variável \rinline{mtcars} disponibiliza dados de várias propriedades de diferentes modelos de automóvel.
	
	\begin{parts}
		\part[2] Calcule a média da potência (\rinline{hp}) dos conjuntos de veículos que têm o mesmo número de mudanças (\rinline{gear}). Não utilize uma estrutura iterativa.~%\makeemptybox{1in}
		
			\begin{solution}
				\begin{rcode}
					> by(mtcars$hp, mtcars$gear, mean)
					mtcars$gear: 3
					[1] 176.1333
					------------------------------------------------------------ 
					mtcars$gear: 4
					[1] 89.5
					------------------------------------------------------------ 
					mtcars$gear: 5
					[1] 195.6
				\end{rcode}
			\end{solution}
			
		\part[1] Visualize um gráfico que represente a relação entre a potência (\rinline{hp}) e o consumo de combustível (\rinline{mpg}). Que conclusão pode tirar ao visualizar esse gráfico?~%\makeemptybox{1in}
		
		\begin{solution}
			\begin{rcode}
				> plot(mtcars$hp, mtcars$mpg)
			\end{rcode}
			O gráfico mostra que maior potência leva a um menor consumo.
		\end{solution}
			
%			\part[2] Nesse \rinline{data.frame}, o consumo é dado em milhas por galão (US). Na Europa, a unidade mais comum para o consumo de combustível é L/100 km. A \autoref{mpgl} mostra uma relação aproximada entre estas duas unidades.
%			
%			\begin{align}
%			\label{mpgl}
%			235 / \text{mpg}_{\text{US}} = \text{L}/100 \text{km}
%			\end{align}
%			
%			Sem recorrer a um ciclo, adicione uma nova coluna à variável \rinline{mtcars} contendo o valor do consumo em L/100.
%			
%			\begin{solution}
%				\begin{rcode}
%					> mtcars2 = transform(mtcars, Consumo= 235/mpg)
%				\end{rcode}
%			\end{solution}
			
			\part[1] Guarde, num vector, os nomes dos modelos que têm a potência abaixo da média e um peso (\rinline{wt}) acima da média.~%\makeemptybox{1in}
			
			\begin{solution}
			\begin{rcode}
			> (lentos = rownames(subset(mtcars, wt > mean(wt) & hp < mean(hp))))
			[1] "Valiant"   "Merc 280"  "Merc 280C"
			\end{rcode}
			\end{solution}
			
			%\pagebreak

			\part[1] Ordene os registos na tabela, colocando primeiro os automóveis mais rápidos. A propriedade \rinline{qsec} contém o tempo necessário para percorrer $\sfrac{1}{4}$ de milha.~%\makeemptybox{1in}
			
			\begin{solution}
				\begin{rcode}
					 mtcars[order(mtcars$qsec), ]
				\end{rcode}
			\end{solution}
			
			\part[1] Guarde a tabela modificada num ficheiro que cumpra as seguintes especificações:
				\begin{itemize}
					\item Valores separados por tabulação;
					\item Vírgula como separador decimal;
					\item Preservar os nomes das colunas e das linhas.
				\end{itemize}~%\makeemptybox{1in}
				
			\begin{solution}
				\begin{rcode}
					write.table(mtcars2, "/Users/aluno/CEGI/mtcars.tsv", sep="\t", dec=",")
				\end{rcode}
			\end{solution}
					
	\end{parts}
	
	\question A distribuição do número de alunos conforme a cor do cabelo, cor dos olhos, e género, está disponível na tabela \rinline{HairEyeColor}.
	
	\begin{parts}
		\part[1] Faça a contagem do número de alunos masculinos e femininos.~%\makeemptybox{1in}
		
		\begin{solution}
			\begin{rcode}
				> margin.table(HairEyeColor,3)
				Sex
				Male Female 
				279    313 
			\end{rcode}
		\end{solution}
		
		\part[2] Construa uma tabela que permita identificar qual é o par de características (cor dos olhos e cor dos cabelos) mais exclusivo nos alunos do sexo masculino, e qual o par de características mais exclusivo nos alunos do sexo feminino.~%\makeemptybox{1in}
		
		\begin{solution}
			\begin{rcode}
				> (h[,,1] - h[,,2])/(h[,,1] + h[,,2])
				Eye
				Hair          Brown       Blue       Hazel      Green
				Black -0.05882353  0.1000000  0.33333333 0.20000000
				Brown -0.10924370  0.1904762 -0.07407407 0.03448276
				Red   -0.23076923  0.1764706  0.00000000 0.00000000
				Blond -0.14285714 -0.3617021  0.00000000 0.00000000
			\end{rcode}
		\end{solution}
		
		%\pagebreak
		
		\part[2] Defina a função \rinline{prob}, que respeita o seguinte resultado.
			\begin{rcode}
				> args(prob)
				function (cabelos, olhos, sexo) 
				NULL
			\end{rcode}
		Esta função, quando chamada, devolve a probabilidade de um aluno, escolhido aleatoriamente do conjunto de dados em questão, ter o cabelo da cor \rinline{cabelos}, a cor dos olhos \rinline{olhos}, e ser do sexo \rinline{sexo}.~%\makeemptybox{1in}
		
		\begin{solution}
			\begin{rcode}
				prob = function(cabelos, olhos, sexo) {
				  HairEyeColor[cabelos, olhos, sexo] / sum(HairEyeColor)
				}
			\end{rcode}
		\end{solution}
		
	\end{parts}
	
	\question[2] Resolva o seguinte sistema de equações.
	
	\begin{align*}
	\left\{\begin{matrix}
	-12.7x & + & 4.2y & = & 1 \\
	3x & - & 10.2y & - & 0.6z & = & 2 \\
	22y & + & z & = & 3
	\end{matrix}\right.
	\end{align*}~%\makeemptybox{1in}
	
	\begin{solution}
		\begin{rcode}
			> coefs = matrix(c(-12.7, 4.2, 0, 3, -10.2, -0.6, 0, 22, 1),
			+ nrow=3, ncol=3, byrow=TRUE)
			> ys = c(1, 2, 3)
			> solve(coefs, ys)
			[1]   0.2556213   1.0110454 -19.2429980
		\end{rcode}
	\end{solution}
	
%	\question[3] Considere agora o conjunto de dados \rinline{iris}, contendo informação sobre 150 amostras de flores, acerca de 4 propriedades e a respectiva espécie.
%	
%	Pretende adicionar uma nova amostra ao conjunto existente. A \autoref{amostra} contém os valores das propriedades dessa amostra. Usando a técnica de classificação do vizinho mais próximo, descubra qual é a sua espécie.
%		
%	\begin{table}[h]
%		\centering
%		\begin{tabulary}{0.9\textwidth}{|C|C|C|C|C|}
%			\hline Número do examplar & Comprimento da Sépala & Largura da Sépala & Comprimento da Pétala & Largura da Pétala \\ 
%			\hline Nova amostra & 6,6 & 3,0 & 3,4 & 2,2 \\  
%			\hline
%		\end{tabulary}
%		\caption{Características da nova amostra de flor Iris.}
%		\label{amostra}
%	\end{table}
%		
%	Assuma que cada característica tem o mesmo peso. Utilize a distância de Mahalanobis, já implementada na função \rinline{mahalanobis}, para encontrar o vizinho mais próximo.
%	
%	\begin{solution}
%		\begin{rcode}
%			> am = c(6.6, 3, 3.4, 2.2)
%			> iris$Species[which.min(mahalanobis(iris[,-5], am, cov(iris[,-5])))]
%			[1] setosa
%		\end{rcode}
%	\end{solution}
	
\end{questions}

\vspace{5mm}
%\centerline
	\begin{center}
		\gradetable[h][questions]
	\end{center}
	
\end{document}