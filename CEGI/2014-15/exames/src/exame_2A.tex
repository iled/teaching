% !TeX spellcheck = pt

%% ************************************************
%% Universidade Nova de Lisboa
%% NOVA Information Management School
%% Computação em Estatística e Gestão de Informação
%% Júlio Caineta, 2015
%% ************************************************
\documentclass[addpoints]{exam}
\usepackage[utf8]{inputenc}
\usepackage{amsmath}
\usepackage[portuguese]{babel}
\usepackage[hidelinks,pdfusetitle]{hyperref}
\author{Júlio Caineta}
\title{CEGI 2014/2015 - Exame 2ª Época - Versão A}
\usepackage{lastpage}
\usepackage{minted}
\usepackage{gensymb}
\usepackage{wrapfig}
\usepackage{tabulary}
\usepackage{graphicx}
\usepackage{xfrac}
\usepackage{multirow}
\usepackage[phantomtext]{dashundergaps}
%\usepackage{color}
%\usepackage{draftwatermark}
%\SetWatermarkText{RASCUNHO}
%\SetWatermarkScale{3}
%\SetWatermarkLightness{0.7}
\cfoot{Página \thepage\ de \pageref{LastPage}}
\renewcommand{\solutiontitle}{\noindent\textbf{Solução:}\par\noindent}
\pointpoints{valor}{valores}
\pointsinrightmargin
\bracketedpoints
\marginpointname{ val.}
\renewcommand*\half{.5}
\hqword{Pergunta}
\hpgword{Página}
\hpword{Cotação}
\hsword{Cotação obtida}
\htword{Total}
\newminted{r}{autogobble}
\newmintinline{r}{}
 
\printanswers
%\shadedsolutions

%\titlegraphic{}

\begin{document}
	
\noindent\begin{minipage}{0.2\textwidth}% adapt widths of minipages to your needs
	\includegraphics[width=2cm]{logo.png}
\end{minipage}
\hfill
\begin{minipage}{0.8\textwidth}
	\begin{center}
		\textsc {\small NOVA IMS -- Universidade Nova de Lisboa} \\
		\textsc {Computação em Estatística e Gestão de Informação -- 2º Semestre 2014/15}
	\end{center}
\end{minipage}

\begin{center}
%	\vspace{5mm} \\
	{\large Exame 2ª Época -- Versão A -- 20/07/2015}
\end{center}
 
\vspace{5mm}
\makebox[\textwidth]{Curso: \underline{\phantom{XXXXX}} Número: \underline{\phantom{XXXXXXXX}} Nome:\enspace\hrulefill}
\vspace{5mm}

{{\large\textbf{Leia, por favor, com atenção:}}
\begin{enumerate}
	\item Este enunciado corresponde à primeira parte do exame (\textbf{R}). O enunciado da segunda parte (SAS) será entregue após o aluno entregar a resolução da primeira parte.
	\item Cada parte está cotada para 10 valores e tem a nota mínima de 4 valores. 
	\item Esta parte deverá ser realizado com acesso a um computador com o programa \textbf{R} instalado. Só poderá aceder ao programa \textbf{R} e ao sistema de ajuda do mesmo.
	\item Os dados necessários para resolver os problemas já se encontram disponíveis na sua instalação do \textbf{R}.
	\item É proibido o uso de qualquer material de apoio (livros, apontamentos, telemóvel), assim como a troca de qualquer informação com os colegas.
	\item Deverá escrever o seu nome, número e curso no cabeçalho desta folha.
	\item As respostas às questões deverão ser dadas, exclusivamente, na folha do enunciado, no espaço reservado para tal. Estas respostas deverão ser código em \textbf{R}.
	\item O não cumprimento de alguma das regras conduzirá à anulação do exame.
	\item A duração do exame, considerando ambas as partes, é de \textbf{2 horas}.
\end{enumerate}

\vspace{10mm}

\begin{questions}
	\question O vector \rinline{precip} contém a precipitação (chuva) média anual, em polegadas (in), de 70 cidades norte americanas.
	
	\begin{parts}
		\part[1] Encontre as cidades, e respectivos valores de precipitação, com os 5\% valores de precipitação média anual mais elevados.%~\makeemptybox{1in}
		
			\begin{solution}
				\begin{rcode}
					> subset(precip, precip >= quantile(precip, 0.95))
					Mobile       Miami New Orleans    San Juan 
					67.0        59.8        56.8        59.2 
				\end{rcode}
			\end{solution}
			
		\part[0\half] Em Portugal, a unidade de medida mais usada para precipitação é o milímetro. Sabendo que 1~in~=~25.4~mm, defina uma função que recebe um vector com valores de precipitação em in, e devolve um vector com valores de precipitação em mm. Converta os valores do vector  \rinline{precip} para mm.%~\makeemptybox{1.3in}
		
		\begin{solution}
			\begin{rcode}
				in2cm = function(precip) {
				  precip * 25.4
				}
				in2cm(precip)
			\end{rcode}
		\end{solution}
		
		\part[1] Utilize a função \rinline{strsplit} para separar os nomes dos elementos do vector \rinline{precip} por palavras (espaços existentes nos nomes). Guarde o resultado numa nova variável denominada \rinline{cidades}.%~\makeemptybox{1.3in}
		
		\begin{solution}
			\begin{rcode}
				cidades = strsplit(names(precip), " ")
			\end{rcode}
		\end{solution}
		
		\part[1\half] Liste os nomes das cidades, existentes no vector \rinline{precip}, cujos nomes são compostos por mais de duas palavras. \textbf{Sugestão:} utilize a lista cidades, criada na alínea anterior.%~\makeemptybox{1.4in}
		
		\begin{solution}
			\begin{rcode}
				> names(precip)[sapply(cidades, function(x) length(x) > 2)]
				[1] "Sault Ste. Marie" "Salt Lake City" 
			\end{rcode}
		\end{solution}
			
	\end{parts}
	
%	\question[0\half] Considere o seguinte conteúdo de um ficheiro de nome \textit{notas.dat}:
%	
%	\begin{minted}{text}
%		Nome-AM2-CEGI-DP1-Est1-ISC-Mrkt
%		Geraldo-12,1-14,4-15,2-11,7-10,0-15,5
%		Justino-8,2-6,6-12,3-9,9-13,1-16,7
%		Felisberta-13,3-13,3-13,3-13,3-13,3-13,3
%		Enzo-18,0-19,0-20,0-17,0-19,0-15,0
%		Toninha-10,0-11,1-12,2-13,3-14,4-15,5
%	\end{minted}
%	
%	Escreva o código necessário para guardar o conteúdo deste ficheiro num \rinline{data.frame} com 5 linhas e 7 colunas.%~\makeemptybox{1in}
%	
%	\begin{solution}
%		\begin{rcode}
%			notas = read.table(notas.dat, header=T, sep="-", dec=",")
%		\end{rcode}
%	\end{solution}
	
	\question O objecto \rinline{Orange} tem 35 linhas e 3 colunas com registos do crescimento de laranjeiras. As variáveis disponibilizadas são:
	
	\begin{description}
		\item[Tree] \rinline{factor} ordenado que indica em que árvore foi efectuada a medição.
		\item[age] Idade da árvore, dada em dias.
		\item[circumference] Perímetro do tronco da árvore, em mm.
	\end{description}
	
	\begin{parts}
		\part[1] Calcule o perímetro médio das laranjeiras, para cada idade registada. Não utilize uma estrutura iterativa.%~\makeemptybox{1.3in}
		
			\begin{solution}
				\begin{rcode}
					by(Orange$circumference, Orange$age, mean)
				\end{rcode}
				ou
				\begin{rcode}
					aggregate(Orange, list(Orange$age), mean)
				\end{rcode}
			\end{solution}
			
\pagebreak

		\part[2] Assuma que a Taxa de Crescimento ($t_c$)  de uma laranjeira é dada pela \autoref{tc}.
		
		\begin{equation}
			\label{tc}
			t_c = \frac{\text{circumference}}{\text{age}}
		\end{equation}
		
		Sem recorrer a um ciclo (explicitamente), calcule a $t_c$ em cada registo. Qual é a árvore com maior Taxa de Crescimento média?%~\makeemptybox{1.3in}
		
		\begin{solution}
			\begin{rcode}
				orange_tc = transform(Orange, tc = circumference / age)
				> which.max(by(orange_tc$tc, orange_tc$Tree, mean))
				4 
				5 
			\end{rcode}
			A árvore com maior Taxa de Crescimento é a 4.
		\end{solution}
	\end{parts}
	
	\question O \rinline{data.frame LifeCycleSavings} contém informações sobre as poupanças das populações, em 50 países, na década de 1960. Considere apenas as colunas \rinline{pop15} e \rinline{pop75}, relativas às percentagens da população abaixo de 15 e acima de 75 anos, respectivamente.
	
	\begin{parts}
		\part[1] O rácio entre \rinline{pop75} e \rinline{pop15} ($t_e$) é um possível indicador de que a população de um país está envelhecida (\autoref{te}). Crie uma nova tabela, contendo apenas os países em que $t_e >\overline{t_e} + \sigma(t_e)$.
		
		\begin{equation}
			\label{te}
			t_e = \frac{\text{pop75}}{\text{pop15}}
		\end{equation}%~\makeemptybox{1.3in}
		
		\begin{solution}
			\begin{rcode}
				> te = LifeCycleSavings$pop75 / LifeCycleSavings$pop15
				> subset(LifeCycleSavings, te > mean(te) + sd(te))
				sr pop15 pop75     dpi ddpi
				Austria        12.07 23.32  4.41 1507.99 3.93
				Belgium        13.17 23.80  4.43 2108.47 3.82
				Denmark        16.85 24.42  3.93 2496.53 3.99
				France         12.64 25.06  4.70 2213.82 4.52
				Germany        12.55 23.31  3.35 2457.12 3.44
				Italy          14.28 24.52  3.48 1390.00 3.54
				Luxembourg     10.35 21.80  3.73 2449.39 1.57
				Norway         10.25 25.95  3.67 2231.03 3.62
				Sweden          6.86 21.44  4.54 3299.49 3.01
				Switzerland    14.13 23.49  3.73 2630.96 2.70
				United Kingdom  7.81 23.27  4.46 1813.93 2.01
			\end{rcode}
		\end{solution}
		
		\part[0\half] Construa um gráfico que permita identificar o mesmo conjunto de países (não é necessário desenhar o gráfico na folha).%~\makeemptybox{1.3in}
		
		\begin{solution}
			\begin{rcode}
				barplot(te)
				abline(h = mean(te) + sd(te), las = 2, space = .5))
			\end{rcode}
			Gráfico de barras com uma linha horizontal que permite identificar a condição requerida. Os dois últimos argumentos são opcionais.
		\end{solution}
	\end{parts}
	
\pagebreak
	
	\question Para resolver os próximos exercícios, considere o seu número mecanográfico (número de aluno).
	
	\begin{parts}
		\part[1] Crie um conjunto de 10 números aleatórios, \rinline{x}, a partir da seguinte distribuição:
		
		\begin{equation*}
			Norm(\mu = xxxx, \sigma= yyy)
		\end{equation*}
		
		em que $xxxx$ são os primeiros quatro dígitos do seu número mecanográfico, e $yyy$ são os restantes três dígitos.%~\makeemptybox{1.4in}
		
		\begin{solution}
			\begin{rcode}
				x = rnorm(n = 10, mean = 1999, sd = 123)
			\end{rcode}
		\end{solution}
				
		\part[0\half] A partir do conjunto criado na alínea anterior, crie um novo conjunto \rinline{y}, com igual número de elementos, através da seguinte linha de código:
		
		\begin{rcode}
			y = sample(x, 10, rep=T)
		\end{rcode}
		
		Verifique se as médias de \rinline{x} e \rinline{y} são iguais e explique porquê.
		
		\textbf{Nota:} se não resolveu a alínea anterior, considere \rinline{x = seq(1234, 4321, length=10)}.%~\makeemptybox{1.4in}
		
		\begin{solution}
			\begin{rcode}
				> y=sample(x, 10, rep=T)
				> mean(x)
				[1] 2007.595
				> mean(y)
				[1] 1995.478
			\end{rcode}
			
			As médias são diferentes porque, apesar dos dois vectores terem o mesmo número de elementos, \rinline{y} foi criado a partir de uma amostragem com reposição, sendo, então, mais provável existirem elementos repetidos.
		\end{solution}
	
	\end{parts}
		
	
\end{questions}

\vspace{5mm}
%\centerline
	\begin{center}
		\gradetable[h][questions]
	\end{center}
	
\end{document}