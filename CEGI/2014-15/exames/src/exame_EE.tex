% !TeX spellcheck = pt_PT-preao

%% ************************************************
%% Universidade Nova de Lisboa
%% NOVA Information Management School
%% Computação em Estatística e Gestão de Informação
%% Júlio Caineta, 2015
%% ************************************************
\documentclass[addpoints]{exam}
\usepackage[utf8]{inputenc}
\usepackage{amsmath}
\usepackage[portuguese]{babel}
\usepackage[hidelinks,pdfusetitle]{hyperref}
\author{Júlio Caineta}
\title{CEGI 2014/2015 - Exame Época Especial}
\usepackage{lastpage}
\usepackage{minted}
\usepackage{gensymb}
\usepackage{wrapfig}
\usepackage{tabulary}
\usepackage{graphicx}
\usepackage{xfrac}
\usepackage{multirow}
\usepackage[phantomtext]{dashundergaps}
%\usepackage{color}
%\usepackage{draftwatermark}
%\SetWatermarkText{RASCUNHO}
%\SetWatermarkScale{3}
%\SetWatermarkLightness{0.7}
\cfoot{Página \thepage\ de \pageref{LastPage}}
\renewcommand{\solutiontitle}{\noindent\textbf{Solução:}\par\noindent}
\pointpoints{valor}{valores}
\pointsinrightmargin
\bracketedpoints
\marginpointname{ val.}
\renewcommand*\half{.5}
\hqword{Pergunta}
\hpgword{Página}
\hpword{Cotação}
\hsword{Cotação obtida}
\htword{Total}
\newminted{r}{autogobble}
\newmintinline{r}{}
 
%\printanswers
%\shadedsolutions

%\titlegraphic{}

\begin{document}
	
\noindent\begin{minipage}{0.2\textwidth}% adapt widths of minipages to your needs
	\includegraphics[width=2cm]{logo.png}
\end{minipage}
\hfill
\begin{minipage}{0.8\textwidth}
	\begin{center}
		\textsc {\small NOVA IMS -- Universidade Nova de Lisboa} \\
		\textsc {Computação em Estatística e Gestão de Informação -- 2º Semestre 2014/15}
	\end{center}
\end{minipage}

\begin{center}
%	\vspace{5mm} \\
	{\large Exame Época Especial -- 25/09/2015}
\end{center}
 
\vspace{5mm}
\makebox[\textwidth]{Curso: \underline{\phantom{XXXXX}} Número: \underline{\phantom{XXXXXXXX}} Nome:\enspace\hrulefill}
\vspace{5mm}

{{\large\textbf{Leia, por favor, com atenção:}}
\begin{enumerate}
	\item Este enunciado corresponde à primeira parte do exame (\textbf{R}). O enunciado da segunda parte (SAS) será entregue após o aluno entregar a resolução da primeira parte.
	\item Cada parte está cotada para 10 valores e tem a nota mínima de 4 valores. 
	\item Esta parte deverá ser realizado com acesso a um computador com o programa \textbf{R} instalado. Só poderá aceder ao programa \textbf{R} e ao sistema de ajuda do mesmo.
	\item Os dados necessários para resolver os problemas já se encontram disponíveis na sua instalação do \textbf{R}.
	\item É proibido o uso de qualquer material de apoio (livros, apontamentos, telemóvel), assim como a troca de qualquer informação com os colegas.
	\item Deverá escrever o seu nome, número e curso no cabeçalho desta folha.
	\item As respostas às questões deverão ser dadas, exclusivamente, na folha do enunciado, no espaço reservado para tal. Estas respostas deverão ser código em \textbf{R}.
	\item O não cumprimento de alguma das regras conduzirá à anulação do exame.
	\item A duração do exame, considerando ambas as partes, é de \textbf{2 horas}.
\end{enumerate}

\vspace{10mm}

\begin{questions}
	\question O vector \rinline{rivers} contém o comprimento, em milhas, dos 141 principais rios norte americanos.
	
	\begin{parts}
		\part[1] Construa um novo vector que contenha o comprimento dos 50 maiores rios, mas considerando apenas os rios numa posição ímpar, aquando da sua ordenação. Ou seja, o rio mais longo, o 3º rio mais longo, o 5º rio mais longo, o 7º mais longo, e assim sucessivamente, até ao máximo de 50 rios.~\makeemptybox{1.2in}
		
		\begin{solution}
			\begin{rcode}
				> sort(rivers, dec=T)[seq(from=1, by=2, length=50)]
				[1] 3710 2348 1885 1459 1306 1243 1171 1054 1000  906  900  870  840  780  735
				[16]  730  720  696  671  630  620  610  600  600  560  540  529  525  505  500
				[31]  470  460  450  444  431  425  420  411  407  390  383  380  375  360  360
				[46]  350  350  340  336  330
			\end{rcode}
		\end{solution}
		
\pagebreak		
		
		\part[1] Escreva a função \rinline{perc}, que receba dois argumentos: um vector numérico, \rinline{v} e um número, \rinline{x}. A função devolve o percentil de \rinline{v} correspondente ao valor de \rinline{x}.
		
		Por exemplo, considerando o vector \rinline{rivers}:
		
			\begin{rcode}
				> perc(rivers, 680)
				[1] 0.751773
				> perc(rivers, 425)
				[1] 0.5035461
				> perc(rivers, 10)
				[1] 0
				> perc(rivers, 10000)
				[1] 1
			\end{rcode}
			
		Estes valores correspondem aos obtidos com a função \rinline{quantile}.
			
			\begin{rcode}
				> quantile(rivers)
				0%    25%    50%    75%    100% 
				135   310    425    680    3710 
			\end{rcode}
		 ~\makeemptybox{1.1in}
		
		\begin{solution}
			\begin{rcode}
				perc = function (v, x) {
					return (length(subset(v, v <= x)) / length(v))
				}
			\end{rcode}
		\end{solution}
		
		\part[1] Suponha que, para uma dado estudo, não interessa considerar os rios com comprimento inferior a 500 km (aproximadamente 311 mi). Utilize a função \rinline{replace} para criar o vector \rinline{rios}, substituindo o valor do comprimento dos referidos rios, por \rinline{NA}.~\makeemptybox{1.1in}
		
		\begin{solution}
			\begin{rcode}
				rios = replace(rivers, rivers < 311, NA)
			\end{rcode}
		\end{solution}
		
		\part[0\half] O coeficiente de variação ($c_v$) é uma medida de dispersão relativa, útil quando duas distribuições têm valores médios ($\mu$) diferentes, caso em que o desvio padrão ($\sigma$) não é comparável (\autoref{cv}).
		
			\begin{equation}
				\label{cv}
				c_v = \frac{\sigma}{\mu}
			\end{equation}
		
		Calcule o coeficiente de variação do vector \rinline{rios}.
		
		Se não resolveu a alínea anterior, considere \rinline{rios = ifelse(rivers > 375, rivers, NA)}. ~\makeemptybox{1.1in}
		
		\begin{solution}
			\begin{rcode}
				> sd(rios, na.rm=T) / mean(rios, na.rm=T)
				[1] 0.746284
			\end{rcode}
		\end{solution}
		
	\end{parts}
	
	
	\question A tabela \rinline{UCBAdmissions} contém informação sobre os candidatos à Universidade da Califórnia em Berkeley (UCB), em 1973, nos seis maiores departamentos. Os dados estão dispostos num objecto tridimensional, cujas variáveis são:
	
	\begin{description}
		\item[1 -- Admit] indica se o candidato foi aceite (\textit{Admitted}) ou não (\textit{Rejected});
		\item[2 -- Gender] género (\textit{Male} / \textit{Female});
		\item[3 -- Dept] departamento que recebeu a candidatura (\textit{A, B, C, D, E, F}).
	\end{description}
	
	\begin{parts}
		\part[1] Encontre o número total de candidaturas rejeitadas. Esse valor corresponde a que percentagem do total de candidaturas?~\makeemptybox{1.3in}
		
			\begin{solution}
				\begin{rcode}
					# para simplificar e facilitar na escrita do código, pode-se criar uma cópia
					# da variável UCBAdmissions com um nome mais curto
					> A = UCBAdmissions
					> (r = sum(A['Rejected', , ]))
					[1] 2771
					> r / sum(A) * 100
					[1] 61.22404
				\end{rcode}
			\end{solution}
		
		\part[1] Nesse ano, a UCB foi acusada de favorecer candidatos do sexo masculino. A taxa de aceitação corresponde ao rácio entre o número de candidatos admitidos e o número de candidatos. Calcule a taxa de aceitação de candidatos separadamente por cada um dos géneros, no total dos seis departamentos.~\makeemptybox{1.3in}
		
			\begin{solution}
				\begin{rcode}
					> (ta.male = sum(A['Admitted', 'Male', ]) / sum(A[, 'Male', ]))
					[1] 0.4451877
					> (ta.female = sum(A['Admitted', 'Female', ]) / sum(A[, 'Female', ]))
					[1] 0.3035422
				\end{rcode}
				
				Nos seis maiores departamentos, houve uma aceitação de 44,5\% dos candidatos masculinos e de 30,4\% dos candidatos femininos.
			\end{solution}
			
		\part[2] Numa análise estatística a um conjunto de dados, pode-se encontrar uma determinada tendência que pode desaparecer ou mudar, quando os mesmos dados são analisados por grupos. Este efeito é denominado por Paradoxo de Simpson. Determine a taxa de aceitação, como na alínea anterior, mas agora separando também por cada um dos departamentos. Na sua resolução, evite repetir código.~\makeemptybox{1.5in}
		
		\begin{solution}
			\begin{rcode}
				ta = function(gen) {
					apply(A, 3, function(t, g) t['Admitted', g] / sum(t[ , g]), gen)
				}
				> ta('Male')
				A          B          C          D          E          F 
				0.62060606 0.63035714 0.36923077 0.33093525 0.27748691 0.05898123 
				> ta('Female')
				A          B          C          D          E          F 
				0.82407407 0.68000000 0.34064081 0.34933333 0.23918575 0.07038123 
			\end{rcode}
			
			Este é um exemplo clássico do Paradoxo de Simpson. Quando se analisa por grupos, verifica-se que, afinal, não houve favorecimento aos candidatos do género masculino. Aconteceu que as candidatas à UCB tentaram, maioritariamente, entrar em departamentos mais competitivos, enquanto que os candidatos optaram, principalmente, por departamentos mais fáceis de entrar.
		\end{solution}
	\end{parts}

\pagebreak
	
	\question[0\half] Considere o seguinte conteúdo de um ficheiro de nome \textit{notas.dat}:
	
	\begin{minted}{text}
		Nome-AM2-CEGI-DP1-Est1-ISC-Mrkt
		Geraldo-12,1-14,4-15,2-11,7-10,0-15,5
		Justino-8,2-6,6-12,3-9,9-13,1-16,7
		Felisberta-13,3-13,3-13,3-13,3-13,3-13,3
		Enzo-18,0-19,0-20,0-17,0-19,0-15,0
		Toninha-10,0-11,1-12,2-13,3-14,4-15,5
	\end{minted}
	
	Escreva o código necessário para guardar o conteúdo deste ficheiro num \rinline{data.frame} com 5 linhas e 7 colunas.~\makeemptybox{0.9in}
	
	\begin{solution}
		\begin{rcode}
			notas = read.table("notas.dat", header=T, sep="-", dec=",")
		\end{rcode}
	\end{solution}
	
	\question O objecto \rinline{ChickWeight} tem 578 linhas e 4 colunas com observações do efeito de diferentes dietas no crescimento de pintainhos. As variáveis disponibilizadas são:
	
	\begin{description}
		\item[weight] vector numérico com o peso de cada pintainho, em grama.
		\item[Time] número de dias passados, desde o nascimento, até à altura da medição do peso.
		\item[Chick] \rinline{factor} ordenado que indica em que pintainho foi efectuada a medição.
		\item[Diet] \rinline{factor} que indica o tipo de dieta (de 1 a 4) aplicado a cada pintainho.
	\end{description}
	
	\begin{parts}
		\part[1] Calcule o peso médio dos pintainhos, para cada instante de tempo registado. Não utilize uma estrutura iterativa.~\makeemptybox{0.9in}
		
			\begin{solution}
				\begin{rcode}
					cw = ChickWeight
					pm = by(cw$weight, cw$Time, mean)
				\end{rcode}
			\end{solution}

		\part[1] Construa um gráfico que mostre o peso médio dos pintainhos ao longo do tempo. Coloque no eixo horizontal o tempo e no eixo vertical o peso. Respeite o intervalo de tempo do ensaio, que vai de 0 a 21 dias.
		
		Se não resolveu a alínea anterior, considere o vector \rinline{pm}, com o peso médio em cada momento de medição:
		
			\begin{rcode}
				pm = c(41, 49, 60, 74, 91, 108, 129, 144, 168, 190, 210, 219)
			\end{rcode}
			
			~\makeemptybox{0.9in}
		
		\begin{solution}
			\begin{rcode}
				plot(x = unique(cw$Time), y=pm, xlab='Tempo', ylab='Peso')
			\end{rcode}
		\end{solution}
	\end{parts}		
	
\end{questions}

\vspace{5mm}
%\centerline
	\begin{center}
		\gradetable[h][questions]
	\end{center}
	
\end{document}