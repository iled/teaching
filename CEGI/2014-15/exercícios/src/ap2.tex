%% ************************************************
%% Universidade Nova de Lisboa
%% NOVA Information Management School
%% Computação em Estatística e Gestão de Informação
%% Júlio Caineta, 2015
%% ************************************************
\documentclass{exam}
\usepackage[utf8]{inputenc}
\usepackage{amsmath}
\usepackage[portuguese]{babel}
\usepackage[hidelinks,pdfusetitle]{hyperref}
\author{Júlio Caineta}
\title{CEGI 2014/2015 - Exercícios 2}
\usepackage{lastpage}
\usepackage{minted}
%\usepackage{color}
\cfoot{Página \thepage\ de \pageref{LastPage}}
\renewcommand{\solutiontitle}{\noindent\textbf{Solução:}\par\noindent}
\newminted{r}{autogobble}
\newmintinline{r}{}
 
%\printanswers
%\shadedsolutions

\begin{document}
 
\begin{center}
	\textsc {\small NOVA IMS -- Universidade Nova de Lisboa} \\
	\textsc {Computação em Estatística e Gestão de Informação -- 2º Semestre 2014/15}
	\vspace{5mm} \\
	{\large Exercícios 2}
\end{center}
 
\vspace{5mm}

\section{Factores}

Em diversos contextos, pode ser necessário lidar com um tipo de dados que não seja numérico, mas que o seu valor possa ser avaliado qualitativamente.

Por exemplo, num inquérito feito aos alunos, considere a questão: ``O que acha dos docentes de CEGI?'', tendo as seguintes hipóteses de escolha ``a) Terríveis'', ``b) \href{http://www.urbandictionary.com/define.php?term=Meh}{Meh.}'', ``c) \href{http://www.urbandictionary.com/define.php?term=uber-cool}{über-cool!}''. Este tipo de dados é então denominado categórico ou nominal.
Neste caso em particular, têm ainda uma relação de ordem: 

\begin{center}
	``Terríveis'' $<$ ``\href{http://www.urbandictionary.com/define.php?term=Meh}{Meh.}'' $<$ ``\href{http://www.urbandictionary.com/define.php?term=uber-cool}{über-cool!}''.
\end{center}

Em \textbf{R}, existe um tipo de dados indicado para estes casos, mais conveniente e vantajoso do que apenas ter um vector de caracteres: \rinline{factor}.
 
\begin{questions}
	
\question Fez-se um levantamento de algumas características fisionómicas dos alunos de uma turma. Crie um vector de factores, \rinline{cor.olhos}, a partir do seguinte vector:

\begin{rcode}
	c("azul", "castanho", "castanho", "castanho", "verde", "azul", "castanho")
\end{rcode}

\begin{solution}
	Para criar um vector de factores, a partir de um vector existente, usa-se a função \rinline{factor}.
		\begin{rcode}
			> cor.olhos = factor(c("azul", "castanho", "castanho", "castanho", "verde",
			+ "azul", "castanho"))
			> cor.olhos
			[1] azul     castanho castanho castanho verde    azul     castanho
			Levels: azul castanho verde
		\end{rcode}
\end{solution}

\question Verifique os níveis (categorias) criados.
\begin{solution}
	
	\begin{rcode}
		> levels(cor.olhos)
		[1] "azul"     "castanho" "verde"   
	\end{rcode}
\end{solution}

\question Um outro conjunto de alunos foi registado no seguinte vector:
\begin{rcode}
	c("castanho", "castanho", "azul")
\end{rcode}
Defina os factores para este vector, mantendo os mesmos níveis.

\begin{solution}
	Neste caso, para forçar que os níveis iniciais sejam respeitados, é necessário especificá-los através do argumento \rinline{levels}.
	\begin{rcode}
		> grupo2 = factor(c("castanho", "castanho", "azul"), levels=levels(cor.olhos))
		> grupo2
		[1] castanho castanho azul    
		Levels: azul castanho verde
	\end{rcode}
\end{solution}

\question Considerando ainda os mesmos níveis, transforme em factores os dados de um terceiro grupo de alunos:

\begin{rcode}
	c("azul", "castanho", "cinzento", "castanho")
\end{rcode}

O que aconteceu ao elemento na 3ª posição?

\begin{solution}
	\begin{rcode}
		> grupo3 = factor(c("azul", "castanho", "cinzento", "castanho"),
		+ levels=levels(cor.olhos))
		> grupo3
		[1] azul     castanho <NA>     castanho
		Levels: azul castanho ver
	\end{rcode}
	A categoria \rinline{"cinzento"} não existe nos níveis que estão a ser impostos, logo o elemento correspondente é convertido para \rinline{NA} no vector de factores.
\end{solution}

\question No vector \rinline{gen} registou-se o género dos alunos:
\begin{rcode}
	gen = c("m", "f", "m", "m", "f", "f", "m")
\end{rcode}

Crie uma tabela de contingência\footnote{\url{https://pt.wikipedia.org/wiki/Tabela_de_conting\%C3\%AAncia}}, mostrando a cor dos olhos por sexo.

\begin{solution}
	\begin{rcode}
		> (tabela = table(gen, cor.olhos))
		cor.olhos
		gen azul castanho verde
		f    1        1     1
		m    1        3     0
	\end{rcode}
\end{solution}

\question Crie duas tabelas de frequências marginais, uma mostrando o número total de homens e mulheres, outra mostrando as contagens pela cor dos olhos.

\begin{solution}
	Este problema tem duas soluções semelhantes. Uma consiste em repetir a solução da questão anterior, passando como argumento apenas um dos vectores. A outra solução, aqui mostrada, parte da tabela resultante da solução anterior.
	\begin{rcode}
		# número total de homens e mulheres
		> margin.table(tabela,1)
		gen
		f m 
		3 4 
		# contagens pela cor dos olhos
		> margin.table(tabela,2)
		cor.olhos
		azul castanho    verde 
		2        4        1 
	\end{rcode}
\end{solution}

\question Seleccione os primeiros três elementos do vector \rinline{cor.olhos}. O que aconteceu aos níveis? Como manter apenas os níveis dos elementos seleccionados?

\begin{solution}
	\begin{rcode}
		> cor.olhos[1:3]
		[1] azul     castanho castanho
		Levels: azul castanho verde
	\end{rcode}
	Ao indexar um ou mais itens de um vector de factores, os níveis são mantidos. Para forçar que sejam considerados apenas os níveis correspondentes aos elementos seleccionados, é necessário usar o argumento \rinline{drop}.
	\begin{rcode}
		> cor.olhos[1:3, drop=TRUE]
		[1] azul     castanho castanho
		Levels: azul castanho
	\end{rcode}
\end{solution}

\question Por fim, guardou-se a informação contendo a altura dos alunos no vector \rinline{alturas}. Esta informação é qualitativa, mas tem relação de ordem.
\begin{rcode}
	alturas = c("baixo", "baixo", "médio", "alto", "alto", "médio", "médio")
\end{rcode}
Transforme em factores, mantendo a relação de ordem. Compare a altura entre dois alunos.

\begin{solution}
	\begin{rcode}
		> alturas = c("baixo", "baixo", "médio", "alto", "alto", "médio", "médio")
		> alturas = factor(alturas, levels=c("baixo", "médio", "alto"), ordered=TRUE)
		> alturas
		[1] baixo baixo médio alto  alto  médio médio
		Levels: baixo < médio < alto
		> alturas[1] > alturas[2]
		[1] FALSE
		> alturas[1] > alturas[3]
		[1] FALSE
		> alturas[1] >= alturas[2]
		[1] TRUE
		> alturas[3] <= alturas[6]
		[1] TRUE
	\end{rcode}
\end{solution}

\end{questions}

\section{Matrizes -- Parte 2}
\label{matrizes}

\begin{questions}
\question Uma matriz é uma extensão de um vector para duas dimensões. Na verdade, em \textbf{R}, um vector não tem dimensão. Considere o seguinte vector, que contém a pauta de uma disciplina:
\begin{rcode}
v = as.integer(rnorm(20, 12, 4))
\end{rcode}
Verifique as suas dimensões, usando a função \rinline{dim}. Transforme-o numa matriz \rinline{pauta}, com 4 linhas e 5 colunas. Verifique agora as dimensões da matriz criada.

\begin{solution}
	\begin{rcode}
		> dim(v)
		NULL
		> pauta = matrix(v, 4, 5)
		> dim(pauta)
		[1] 4 5
	\end{rcode}
\end{solution}

\question Junte nomes às linhas e às colunas da matriz \rinline{pauta}, em dois passos distintos.

\begin{solution}
	\begin{rcode}
		> rownames(pauta) = paste(rep("Aluno", 4), 1:4)
		> colnames(pauta) = letters[1:5]
		> pauta
		         a  b  c  d  e
		Aluno 1  5  8  6  6 21
		Aluno 2 13 15 12 19 11
		Aluno 3 19 15 13 14  8
		Aluno 4 11 11  4  8 13
	\end{rcode}
\end{solution}

\question Aceda aos elementos da matriz de diferentes formas:
\begin{itemize}
\item duas posições
\item dois nomes
\item uma posição e um nome
\item uma linha inteira pela posição
\item uma coluna inteira pelo nome
\item primeiras 3 linhas, todas as colunas excepto a última
\end{itemize}

\begin{solution}
	\begin{rcode}
		# 2ª linha, 3ª coluna
		> pauta[2, 3]
		[1] 12
		# linha "Aluno 3", coluna "d"
		> pauta["Aluno 3", "d"]
		[1] 14
		# 4ª linha, coluna "a"
		> pauta[4, "a"]
		[1] 11
		# 3ª linha completa
		> pauta[3, ]
		a  b  c  d  e 
		19 15 13 14  8 
		# coluna "e" completa
		> pauta[ , "e"]
		Aluno 1 Aluno 2 Aluno 3 Aluno 4 
		21      11       8      13 
		# primeiras 3 linhas, todas as colunas excepto a última (5ª)
		> pauta[1:3, -5]
		         a  b  c  d
		Aluno 1  5  8  6  6
		Aluno 2 13 15 12 19
		Aluno 3 19 15 13 14
	\end{rcode}
\end{solution}

\question Junte uma nova coluna à matriz. Verifique agora os nomes das colunas da matriz. Como adicionar um nome à coluna recentemente adicionada?

\begin{solution}
	\begin{rcode}
		> pauta = cbind(pauta, 10:13)
		# a coluna é acrescentada mas fica sem nome atribuído
		> pauta
		         a  b  c  d  e   
		Aluno 1  5  8  6  6 21 10
		Aluno 2 13 15 12 19 11 11
		Aluno 3 19 15 13 14  8 12
		Aluno 4 11 11  4  8 13 13
		> colnames(pauta)
		[1] "a" "b" "c" "d" "e" "" 
		> colnames(pauta)[6]
		[1] ""
		# aceder ao nome dessa coluna e modificá-lo
		> colnames(pauta)[6] = "f"
		> pauta
		         a  b  c  d  e  f
		Aluno 1  5  8  6  6 21 10
		Aluno 2 13 15 12 19 11 11
		Aluno 3 19 15 13 14  8 12
		Aluno 4 11 11  4  8 13 13
	\end{rcode}
\end{solution}

\question Junte uma nova linha à matriz, atribuindo-lhe logo um nome.
\begin{solution}
	\begin{rcode}
		> pauta = rbind(pauta, "Aluno 5"=sample(v,6))
		> pauta
		         a  b  c  d  e  f
		Aluno 1  5  8  6  6 21 10
		Aluno 2 13 15 12 19 11 11
		Aluno 3 19 15 13 14  8 12
		Aluno 4 11 11  4  8 13 13
		Aluno 5  6 19  5 21  8  8
	\end{rcode}
	Nota: a função \rinline{sample} está a ser usada apenas para obter um conjunto de notas aleatoriamente, a partir do vector \rinline{v}.
\end{solution}

\question Calcule a média dos valores em cada linha e em cada coluna.
\begin{solution}
	\begin{rcode}
		# por linha
		> rowMeans(pauta)
		  Aluno 1   Aluno 2   Aluno 3   Aluno 4   Aluno 5 
		 9.333333 13.500000 13.500000 10.000000 11.166667 
		# por coluna
		> colMeans(pauta)
		   a    b    c    d    e    f 
		10.8 13.6  8.0 13.6 12.2 10.8 
	\end{rcode}
\end{solution}

\question Crie o vector \rinline{final}, em que cada elemento tem o valor médio da linha correspondente, mas apenas no caso deste valor ser positivo ($\geq 10$), caso contrário, introduza o valor especial \rinline{NA}. Junte o vector \rinline{final} à matriz \rinline{pauta}.
\begin{solution}
	\begin{rcode}
		> final = ifelse(rowMeans(pauta) >= 10, rowMeans(pauta), NA)
		> pauta = cbind(pauta, final)
		> pauta
		         a  b  c  d  e  f    final
		Aluno 1  5  8  6  6 21 10       NA
		Aluno 2 13 15 12 19 11 11 13.50000
		Aluno 3 19 15 13 14  8 12 13.50000
		Aluno 4 11 11  4  8 13 13 10.00000
		Aluno 5  6 19  5 21  8  8 11.16667
	\end{rcode}
\end{solution}

\question Experimente avaliar o resultado de algumas operações aritméticas sobre a seguinte matriz:
\begin{rcode}
	ma = matrix(as.integer(runif(16, 1, 20)), 4, 4)
\end{rcode}
Exemplos:
\begin{itemize}
\item \rinline{ma + 1}
\item \rinline{ma * 2}
\item \rinline{ma * ma}
\item \rinline{ma \%*\% ma}
\item determinante, com a função \rinline{det}
\item transposta, com a função \rinline{t}
\item inverter, com a função \rinline{solve}
\end{itemize}

\begin{solution}
	\begin{rcode}
		> ma = matrix(as.integer(runif(16, 1, 20)), 4, 4)
		> ma
		      [,1] [,2] [,3] [,4]
		[1,]   11   13   14    1
		[2,]   19    8   16   11
		[3,]   11    5   18   18
		[4,]   11    1   14    6
		> ma + 1
		     [,1] [,2] [,3] [,4]
		[1,]   12   14   15    2
		[2,]   20    9   17   12
		[3,]   12    6   19   19
		[4,]   12    2   15    7
		> ma * 2
		     [,1] [,2] [,3] [,4]
		[1,]   22   26   28    2
		[2,]   38   16   32   22
		[3,]   22   10   36   36
		[4,]   22    2   28   12
		> ma %*% ma
		     [,1] [,2] [,3] [,4]
		[1,]  533  318  628  412
		[2,]  658  402  836  461
		[3,]  612  291  810  498
		[4,]  360  227  506  310
		> det(ma)
		[1] 16476
		> t(ma)
		     [,1] [,2] [,3] [,4]
		[1,]   11   19   11   11
		[2,]   13    8    5    1
		[3,]   14   16   18   14
		[4,]    1   11   18    6
		> solve(ma)
		            [,1]        [,2]        [,3]          [,4]
		[1,] -0.04685603  0.12066035 -0.07137655  0.0007283321
		[2,]  0.06724933  0.01335276  0.02731245 -0.1176256373
		[3,]  0.04855547 -0.10949260  0.02603787  0.1145302258
		[4,] -0.03860160  0.03204661  0.06554989 -0.0823015295
		
	\end{rcode}
\end{solution}

\question Resolva o seguinte sistema de equações em \textbf{R}:
\begin{align*}
\left\{\begin{matrix}
-4x & + & 0.3y & = & 12.3 \\
54.3x & - & 4y & = & 45
\end{matrix}\right.
\end{align*}

\begin{solution}
	\begin{rcode}
		> coefs = matrix(c(-4, 0.3, 54.3, -4), nrow=2, ncol=2, byrow=TRUE)
		> ys = c(12.3, 45)
		> solve(coefs, ys)
		[1]  216.2069 2923.7586
	\end{rcode}
\end{solution}

\end{questions}

\section{Arrays}
Em \textbf{R}, as matrizes são uma extensão dos vectores para duas dimensões. Os arrays aumentam essa extensão para qualquer número de dimensões. Seguem as mesmas regras e possibilidades de indexação que os vectores e matrizes.

\begin{questions}
\question Crie um array com 50 elementos, distribuídos por 2 linhas, 3 colunas, e 5 camadas.

\begin{solution}
	\begin{rcode}
		array(1:50, dim=c(2, 3, 5))
	\end{rcode}
\end{solution}


\question Considere a matriz \rinline{pauta} da secção anterior. Junte várias cópias dessa matriz num array, como se, por exemplo, estivesse a agregar as pautas de diferentes anos, num só objecto.

\begin{solution}
	\begin{rcode}
		# criar cópias da matriz pauta, ligeiramente diferentes, só para ilustrar
		# com valores diferentes
		pauta2013 = pauta - 1
		pauta2014 = pauta - 2
		pauta2015 = pauta + 2
		pautas = array(c(pauta2013, pauta2014, pauta2015), dim=c(5, 7, 3))
	\end{rcode}
\end{solution}

\question Experimente aceder a elementos do array do mesmo modo que tentou para a matriz.

\begin{solution}
	\begin{rcode}
		# notas de todos os alunos, na 2ª componente, em todos os anos
		> pautas[, 2, ]
		     [,1] [,2] [,3]
		[1,]    7    6   10
		[2,]   14   13   17
		[3,]   14   13   17
		[4,]   10    9   13
		[5,]   18   17   21
		# notas dos primeiros 3 alunos, em todas as componentes, ignorando o primeiro ano
		> pautas[1:3, , -1]
		, , 1
		
		     [,1] [,2] [,3] [,4] [,5] [,6] [,7]
		[1,]    3    6    4    4   19    8   NA
		[2,]   11   13   10   17    9    9 11.5
		[3,]   17   13   11   12    6   10 11.5
		
		, , 2
		
		     [,1] [,2] [,3] [,4] [,5] [,6] [,7]
		[1,]    7   10    8    8   23   12   NA
		[2,]   15   17   14   21   13   13 15.5
		[3,]   21   17   15   16   10   14 15.5
	\end{rcode}
\end{solution}

\end{questions}

\section{Listas}

As listas, ao contrário dos vectores, matrizes e arrays, permitem guardar elementos de diferentes tipos (e.g., \rinline{"character"},\mintinline{r}{"numeric"} e \rinline{"logical"}) num único objecto. Assim, podem ser vistas como uma colecção ordenada de objectos.

Os objectos que constituem uma matriz são designados por componentes, são numerados, e podem ter um nome associado.

Considere, como exemplo, a lista \rinline{estudante}, que corresponde a uma ficha com os seguintes dados de um aluno:

\begin{center}
\begin{description}
\item[nro] 1999001
\item[nome] John Doe
\item[notas] 14, 3, 12, 10, 19
\item[inscrito] Sim
\end{description}
\end{center}

\begin{questions}
\question Crie a lista \rinline{estudante} com as informações do aluno.
\begin{solution}
	\begin{rcode}
		 estudante = list(nro=1999001, nome="John Doe", notas=c(14, 3, 12, 10, 19),
		   inscrito=TRUE)
	\end{rcode}
\end{solution}

\question Visualize todos os componentes da lista. Compare com o resultado da função \rinline{str}, quando chamada com a mesma lista.
\begin{solution}
	\begin{rcode}
		> estudante
		$nro
		[1] 1999001
		
		$nome
		[1] "John Doe"
		
		$notas
		[1] 14  3 12 10 19
		
		$inscrito
		[1] TRUE
		
		> str(estudante)
		List of 4
		$ nro     : num 2e+06
		$ nome    : chr "John Doe"
		$ notas   : num [1:5] 14 3 12 10 19
		$ inscrito: logi TRUE
	\end{rcode}
	A função \rinline{str} (\textit{structure}) apresenta o conteúdo de um objecto de uma forma mais compacta e fácil de ler.
\end{solution}

\question Aceda ao primeiro componente da lista, usando a mesma sintaxe que nos vectores. Verifique o tipo de objecto desse resultado, usando a função \rinline{typeof}.

\begin{solution}
	\begin{rcode}
		> estudante[1]
		$nro
		[1] 1999001
		
		> typeof(estudante[1])
		[1] "list"
		> estudante[[1]]
		[1] 1999001
		> typeof(estudante[[1]])
		[1] "double"
	\end{rcode}
	Na primeira forma de indexação, o resultado é uma sublista contendo apenas a primeira componente. Na segunda forma, com duplos parêntesis rectos, a indexação resulta no acesso ao conteúdo da primeira componente.
\end{solution}

\question Corrija o nome do primeiro componente para \rinline{"número"}.

\begin{solution}
	\begin{rcode}
		names(estudante)[1] = "número"
	\end{rcode}
\end{solution}

\question Recebeu mais um dado do aluno: tem 33 anos de idade. Actualize a sua ficha com essa informação.

\begin{solution}
	\begin{rcode}
		estudante$idade = 33
	\end{rcode}
\end{solution}


\question Acrescente ainda a seguinte informação à ficha do aluno, em apenas um passo:

\begin{description}
\item[sexo] masculino
\item[pais] "Maria" e "Zé"
\end{description}

\begin{solution}
	\begin{rcode}
		estudante = c(estudante, list(sexo="m", pais=c("Maria", "Zé")))
	\end{rcode}
\end{solution}

\question O aluno recebeu a nota de mais um exame: 10 val. Adicione essa nota às restantes notas.

\begin{solution}
	\begin{rcode}
		estudante$notas = c(estudante$notas, 10)
	\end{rcode}
\end{solution}

\question Quantos componentes já tem a ficha do aluno?

\begin{solution}
	\begin{rcode}
		> length(estudante)
		[1] 7
	\end{rcode}
\end{solution}

\end{questions}

\section{Data Frames}

A estrutura de dados \rinline{data.frame} é das mais usadas em \textbf{R}, em boa parte, devido à sua versatilidade no tipo de objectos suportado e na indexação: é um tipo de dados derivado das listas, suportando elementos de diferentes tipo num só objecto; e adiciona as funcionalidades de indexação dos vectores. 

É mais usado para guardar tabelas de dados, sendo parecido com uma matriz com nomes nas linhas e nas colunas, mas suportando elementos de diferentes tipos. Este tipo de tabela pode ser visto como uma tabela de uma base de dados, em que cada linha corresponde a um registo diferente.

\begin{questions}
\question Transforme a matriz \rinline{pauta}, criada no grupo \ref{matrizes}, num \rinline{data.frame}, e guarde-o na variável \rinline{pautadf}. \textbf{Nota:} não considere a coluna \rinline{final}, adicionada posteriormente.

\begin{solution}
	\begin{rcode}
		pautadf = data.frame(pauta)
	\end{rcode}
\end{solution}

\question Repita o exercício 2.3, agora aplicado à \rinline{pautadf}.

\begin{solution}
	A solução desta alínea é idêntica à do exercício 2.3.
\end{solution}

\question Determine o número de alunos (linhas) e o número de componentes de avaliação (colunas). No exercício 2.1 referiu-se a função \rinline{dim}. Procure por uma forma alternativa de encontrar o número de linhas e de colunas no ecrã de ajuda da função \rinline{dim}.

\begin{solution}
	\begin{rcode}
		# número de linhas
		> nrow(pautadf)
		[1] 5
		# número de colunas
		> ncol(pautadf)
		[1] 6
	\end{rcode}
\end{solution}

\question Considerando que cada linha corresponde a um aluno diferente, e que cada coluna corresponde a uma componente de avaliação, mostre apenas os alunos que tiveram positiva em todas as componentes.

\begin{solution}
	Este exercício pode ser resolvido de várias maneiras. Uma delas, pode ser recorrendo a uma instrução iterativa:
	\begin{rcode}
		aluno = 1
		while (aluno <= nrow(pautadf)) {
		  if (all(pautadf[aluno, ] >= 10)) {
		    print(rownames(pautadf)[aluno])
		  }
		  aluno = aluno + 1
		}
		[1] "Aluno 2"
	\end{rcode}
	No entanto, sendo o \textbf{R} uma linguagem versátil e compatível com o paradigma de programação funcional, é possível resolver este exercício de uma forma mais curta e eficiente (além de elegante):
	\begin{rcode}
		> rownames(pautadf)[apply(pautadf, 1, function(x) all(x >= 10))]
		[1] "Aluno 2"
	\end{rcode}
\end{solution}

\question Mostre as componentes em que pelo menos 3 alunos tiveram positiva. \textbf{Dica:} utilize a coerção de valores lógicos para valores numéricos.

\begin{solution}
	À semelhança do exercício anterior, também este é possível resolver das duas formas referidas.
	
	Versão imperativa:
	\begin{rcode}
		for (componente in 1:ncol(pautadf)) {
		  if (sum(pautadf[, componente] >= 10) >= 3) {
		    print(colnames(pautadf)[componente])
		  }
		}
		[1] "a"
		[1] "b"
		[1] "d"
		[1] "e"
		[1] "f"
	\end{rcode}
	Versão funcional:
	\begin{rcode}
		> colnames(pautadf)[apply(pautadf, 2, function(x) sum(x >= 10) >= 3)]
		[1] "a" "b" "d" "e" "f"
	\end{rcode}
\end{solution}

\question Acrescente uma nova coluna à \rinline{pautadf}, em que cada elemento deve tomar o valor \rinline{"Aprovado"} ou o valor \rinline{"Reprovado"}, conforme a média das notas de cada componente seja positiva ($\geq 10$) ou não. \textbf{Dica:} veja como resolveu o exercício 2.7.

\begin{solution}
	\begin{rcode}
		pautadf$estado = ifelse(rowMeans(pautadf) >= 10, "Aprovado", "Reprovado")
	\end{rcode}
\end{solution}

\question Os alunos foram aleatoriamente distribuídos em dois grupos, conforme mostra o bloco de código em baixo. Explique-o da melhor maneira que conseguir.
\begin{rcode}
	Grupos = sample(paste("Grupo",1:2), nrow(pautadf), replace=T)
	pautadf = data.frame(pautadf, Grupos)
\end{rcode}

\begin{solution}
	A solução deste exercício fica para discussão entre os alunos.
\end{solution}

\question Depois da execução das instruções dadas na alínea anterior, uma nova coluna surge no data frame \rinline{pautadf}. Verifique o tipo de objecto com que os elementos dessa coluna ficaram. O que aconteceu?

\begin{solution}
	\begin{rcode}
		> typeof(pautadf$Grupos)
		[1] "integer"
		> class(pautadf$Grupos)
		[1] "factor"
	\end{rcode}
	A criação de um \rinline{data.frame} com uma coluna cujos elementos sejam do tipo \rinline{character}, converte essa coluna automaticamente para \rinline{factor}.
\end{solution}

\question Utilize o editor interactivo, disponibilizado pela função \rinline{edit}, para alterar alguns valores (à sua escolha) do data frame.

\begin{solution}
	\begin{rcode}
		pautadf = edit(pautadf)
	\end{rcode}
	Com o editor interactivo, é possível editar o conteúdo das células de uma tabela, como se fosse uma folha de cálculo.
\end{solution}
	
\end{questions}
\end{document}