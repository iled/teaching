%% ************************************************
%% Universidade Nova de Lisboa
%% NOVA Information Management School
%% Computação em Estatística e Gestão de Informação
%% Júlio Caineta, 2015
%% **********************************************
\documentclass{exam}
\usepackage[utf8]{inputenc}
\usepackage[portuguese]{babel}
\usepackage[hidelinks,pdfusetitle]{hyperref}
\author{Júlio Caineta}
\title{CEGI 2014/2015 - Exercícios 1}
\usepackage{lastpage}
\usepackage{minted}
\usepackage{paralist}
%\usepackage{color}
\cfoot{Página \thepage\ de \pageref{LastPage}}
\renewcommand{\solutiontitle}{\noindent\textbf{Solução:}\par\noindent}
 
%\printanswers
%\shadedsolutions

\begin{document}
 
\begin{center}
\textsc {\small NOVA IMS -- Universidade Nova de Lisboa} \\
\textsc {Computação em Estatística e Gestão de Informação -- 2º Semestre 2014/15}
\vspace{5mm} \\
{\large Exercícios 1}
\end{center}
 
\vspace{5mm}

\section{Expressões}
Uma expressão em \textbf{R} corresponde a uma sequência de operações, e que pode incluir chamadas de variáveis e de funções, que poderá ser avaliada, tendo um (e apenas um) retorno.
 
\begin{questions}
\question  Avalie as seguintes expressões: \
\begin{itemize}
\item \mintinline{r}{123 * 321}
\item \mintinline{r}{42 ** 2}
\item \mintinline{r}{42 ^ 2}
\item \mintinline{r}{34 - 2 * 6}
\item \mintinline{r}{4 + 2 == 42}
\item \mintinline{r}{1 + sum(2, 3)}
\end{itemize}

\begin{solution}
\begin{minted}{R}
> 123 * 123
[1] 15129
# o expoente pode ser obtido com os operadores ** e ^ (acento circunflexo)
> 42 ** 2
[1] 1764
> 42 ^ 2
[1] 1764
> 34 - 2 * 6
[1] 22
# exemplo com o operador lógico de igualdade
> 4 + 2 == 42
[1] FALSE
# uma expressão pode envolver a chamada de uma função
> 1 + sum(2, 3)
[1] 6
\end{minted}
Notas:
\begin{itemize}
\item Para inserir um comentário no código, ou comentar uma instrução, basta colocar um cardinal \texttt{\#} antes do texto pretendido;
\item O \textbf{R} respeita a precedência dos operadores aritméticos, bem como de outros operadores próprios, pelo que deve-se ter cuidado com a ordem das operações, e usar parêntesis caso seja necessário;
\end{itemize}
\end{solution}

\question Guarde o resultado de uma expressão numa variável.
\begin{solution}
\begin{minted}{R}
> (w = 1 + sum(2, 3))
[1] 6
\end{minted}
Notas:
\begin{itemize}
\item Em \textbf{R}, existem três operadores de atribuição: \mintinline{R}{=}, \mintinline{R}{<-} e \mintinline{R}{->}, pode-se usar qualquer um deles, mas convém ser consistente na sua utilização durante a escrita do código;
\item Neste exemplo, como a variável está colocada à esquerda, só se poderia utilizar os dois primeiros operadores de atribuição;
\item Colocando uma expressão de atribuição entre parêntesis é uma forma rápida de guardar o resultado e visualizá-lo de seguida, sem ter de se escrever o nome da variável.
\end{itemize}
\end{solution}

\question Use essa variável numa outra expressão.
\begin{solution}
\begin{minted}{R}
> x = w + 2
> x
[1] 8
\end{minted}
Guardar o resultado de uma expressão, que pode ser complicada, numa variável auxiliar, pode ser uma forma de evitar repetição de código e de o tornar mais legível.
\end{solution}

\question Atribua, à mesma variável, o número 42. O que aconteceu ao resultado da expressão que havia sido lá guardado?
\begin{solution}
\begin{minted}{R}
> w = 42
> w
[1] 42
\end{minted}
A operação de atribuição é destrutiva, isto é, se for atribuído um novo valor a uma variável existente, o valor anterior deixa de ser acessível. 
\end{solution}
\end{questions}

\section{Vectores -- Parte 1}

Vectores constituem uma forma simples, mas poderosa, de armazenar informação em \textbf{R}, mas apenas de um só tipo. Considere a pauta mostrada na \autoref{tab: pauta}.
\begin{table}[htdp]
\caption{Pauta de CEGI}
\label{tab: pauta}
\begin{center}
\begin{tabular}{|c|c|}
\hline
Nome & Nota \\ \hline
André & 10 \\
Carolina & 12 \\
João & -- \\
Mariana & 15 \\
Pedro & 16 \\
Vânia & 13 \\
Zé & 7 \\
\hline
\end{tabular}
\end{center}
\label{pauta}
\end{table}
\begin{questions}
\question Guarde a tabela em dois vectores, \mintinline{r}{nomes} e \mintinline{r}{notas}. Atente à nota em falta. O que acontecerá se colocar todos os dados num só vector?
\begin{solution}
	\begin{minted}{R}
nomes = c("André", "Carolina", "João", "Mariana", "Pedro", "Vânia", "Zé")
notas = c(10, 12, NA, 15, 16, 13, 7)
	\end{minted}
	Os dados em falta devem ser representados pelo valor especial \mintinline{R}{NA}.
	
	Se colocar todos os dados num só vector, terá duas desvantagens:
	\begin{inparaenum}[\itshape a\upshape)]
		\item torna-se mais difícil aceder aos seus elementos, pois perde-se a \textit{individualidade} das colunas da tabela; e
		\item o tipo de objecto dos valores correspondentes às notas é automaticamente convertido de \mintinline{r}{numeric} para \mintinline{r}{character}, uma vez que um vector apenas suporta um único tipo de objecto de cada vez.
	\end{inparaenum}
\end{solution}

\question Verifique, com um operador lógico, se os dois vectores têm o mesmo tamanho.
\begin{solution}
	\begin{minted}{R}
> length(nomes) == length(notas)
[1] TRUE
	\end{minted}
\end{solution}

\question Aceda ao nome e à nota do \textbf{5º} aluno.
\begin{solution}
	\begin{minted}{R}
> nomes[5]
[1] "Pedro"
> notas[5]
[1] 16
	\end{minted}
\end{solution}

\question Corrigiu-se o teste do João, ele teve 8. Atribua-lhe, no vector \mintinline{r}{notas}, a sua cotação.
\begin{solution}
	Sabendo que o João está colocado na 3ª posição dos vector \mintinline{r}{notas}, basta aceder ao elemento do vector nessa posição.
	\begin{minted}{R}
notas[3] = 8
> notas
[1] 10 12  8 15 16 13  7
	\end{minted}
	Alternativamente, é possível alterar o vector \mintinline{r}{notas} sem saber qual a posição correspondente à nota do João, tirando essa informação a partir do vector \mintinline{r}{nomes}, com uma operação booleana. Atenção que esta solução só é válida se não existirem dois alunos com o mesmo nome.
	\begin{minted}{r}
notas[nomes == "João"] = 8
	\end{minted}
	Outra solução, mais genérica, pode ser obtida recorrendo à função \mintinline{r}{is.na}. Esta função devolve o valor lógico \mintinline{r}{TRUE} se o seu argumento for \mintinline{r}{NA}. Aplicada num vector, o seu retorno é também um vector.
	\begin{minted}{r}
notas[is.na(notas)] = 8
	\end{minted}
	Esta solução tem a vantagem de encontrar um qualquer número de alunos sem nota atribuída. Se o objectivo fosse apenas saber em que posições do vector \mintinline{r}{notas} ocorrem valores em falta, \mintinline{r}{NA}, a solução passaria por usar a função \mintinline{r}{which}.
	\begin{minted}{r}
> which(is.na(notas))
[1] 3
	\end{minted}
\end{solution}

\question Calcule a média das notas.
\begin{solution}
	\begin{minted}{R}
> mean(notas)
[1] 11.57143
	\end{minted}
\end{solution}

\question O docente responsável decidiu atribuir bónus aos alunos. Some uma cotação de bonificação a cada nota.
\begin{solution}
	Supondo que o valor de bónus é de 1 valor, e que será atribuído a todos os alunos:
	\begin{minted}{R}
notas = notas + 1
> notas
[1] 11 13  9 16 17 14  8
	\end{minted}
	O valor 1 foi somado a cada um dos elementos do vector \mintinline{r}{notas}.
	
	Imagine, agora, que o valor de bónus pode ser de 1 ou 0,5 valores. Desprezando a forma como é decidida a atribuição de uma destas hipóteses aos alunos, uma solução seria:
	\begin{minted}{R}
> notas + c(1, 0.5)
[1] 11.0 12.5  9.0 15.5 17.0 13.5  8.0
Warning message:
In notas + c(1, 0.5) :
  longer object length is not a multiple of shorter object length
	\end{minted}
	Repare que os elementos do vector \mintinline{r}{c(1, 0.5)} foram repetidos tantas vezes quantas as necessárias, até serem percorridas todos os elementos do vector \mintinline{r}{notas}. A mensagem de aviso dada pelo \textbf{R} informa que os dois vectores, \mintinline{r}{notas} e \mintinline{r}{c(1, 0.5)}, não têm comprimentos múltiplos (o primeiro tem 7 elementos, o outro tem apenas 2). Por esse motivo, o segundo elemento do vector menor foi repetido menos uma vez.
	
	Estas duas soluções mostram, assim, exemplos da \textit{regra da reciclagem}.
\end{solution}

\end{questions}

\section{Sequências}

Outra forma fácil e útil de criar vectores é através de sequências. Estas podem ser definidas ou até serem aleatórias.

\begin{questions}
\question Crie as seguintes sequências:
\begin{itemize}
\item de 1 a 20
\item de 1 a 20, mas de 2 em 2
\item 15 elementos de 1 a 20
\item 20 elementos a começar em 1, espaçados por 0,4
\item o número 1 repetido 100 vezes
\end{itemize}
\begin{solution}
	\begin{minted}{R}
> # de 1 a 20
> 1:20
[1]  1  2  3  4  5  6  7  8  9 10 11 12 13 14 15 16 17 18 19 20
> # de 1 a 20, mas de 2 em 2
> seq(1, 20, 2)
[1]  1  3  5  7  9 11 13 15 17 19
# 15 elementos de 1 a 20
> seq(1, 20, length=15)
 [1]  1.000000  2.357143  3.714286  5.071429  6.428571  7.785714  9.142857
 [8] 10.500000 11.857143 13.214286 14.571429 15.928571 17.285714 18.642857
[15] 20.000000
# 20 elementos a começar em 1, espaçados por 0,4
> seq(from=1, length=20, by=0.4)
[1] 1.0 1.4 1.8 2.2 2.6 3.0 3.4 3.8 4.2 4.6 5.0 5.4 5.8 6.2 6.6 7.0 7.4 7.8 8.2
[20] 8.6
# o número 1 repetido 100 vezes
> rep(1, 100)
 [1] 1 1 1 1 1 1 1 1 1 1 1 1 1 1 1 1 1 1 1 1 1 1 1 1 1 1 1 1 1 1 1 1 1 1 1 1 1
[38] 1 1 1 1 1 1 1 1 1 1 1 1 1 1 1 1 1 1 1 1 1 1 1 1 1 1 1 1 1 1 1 1 1 1 1 1 1
[75] 1 1 1 1 1 1 1 1 1 1 1 1 1 1 1 1 1 1 1 1 1 1 1 1 1 1
	\end{minted}
	Para uma sequência simples, com números inteiros consecutivos, em ordem crescente ou decrescente, basta usar o operador \mintinline{r}{:}.
	
	Para uma maior manipulação da sequência, a função \mintinline{r}{seq} é mais indicada. Esta função admite vários argumentos, sendo os mais importantes:
	\textbf{from} (a partir de), \textbf{to} (até), \textbf{by} (incremento), e \textbf{length} (número de elementos).
	
	A função \mintinline{r}{rep} também permite criar uma sequência a partir da repetição de outros objectos de tipo básico, além do tipo \mintinline{r}{numeric} (e.g., \mintinline{r}{character}).
\end{solution}

\question Em \textbf{R}, é fácil criar sequências aleatórias, partindo de diferentes distribuições estatísticas. As funções que criam essas sequências têm a forma \textbf{r\textit{func}}, por exemplo, \mintinline{r}{rnorm}, \mintinline{r}{runif} e \mintinline{r}{rt}. Crie as seguintes sequências aleatórias:
\begin{itemize}
\item 10 elementos de uma distribuição normal padrão
\item 10 elementos de uma distribuição normal com média 13 e desvio padrão 2
\item 100 elementos de uma distribuição uniforme entre 1 e 20
\item 5 elementos de uma distribuição \textit{t-student} com 10 graus de liberdade
\end{itemize}
\begin{solution}
	\begin{minted}{R}
	> rnorm(10)
	[1]  0.3893928 -0.8923915  1.4273738 -0.8857474  0.9354829  1.3247520
	[7] -0.8557412 -0.2787961 -1.2497442  0.6856208
	> rnorm(10, mean=13, sd=2)
	[1] 12.81522 13.64655 13.47899 10.56444 15.53645 13.72218 11.47630 12.50179
	[9] 14.89978 12.98178
	> runif(100, 1, 20)
	[1] 12.890452 13.809040  1.845204  3.036405 10.317447  2.687971 18.951556
	[8] 10.019654  3.862149 18.049652 17.856375  2.917227 12.797862 13.444384
	[15]  1.453109  6.730706  9.319316 14.636486 12.255608  1.659550  1.907945
	[22] 13.974350  6.655288  3.489309 16.469166  2.655856  6.802787 18.126680
	[29] 12.420356 15.783056  9.551025 18.970067  2.717150 16.061868 12.821729
	[36] 14.888721  7.398816 19.788180  7.951151 16.325348  2.979728 18.666736
	[43]  6.958047 12.265295 14.980399 13.104052  3.292474 19.787362 12.883868
	[50]  3.825523 12.087716 15.161680 17.281745 17.906548 13.996133 10.566860
	[57] 10.671736 13.204635  7.980697 15.047350  5.762803  9.393275  8.774348
	[64]  9.017975 14.615049 16.550216 12.790073 16.638168  5.216438 11.186398
	[71] 13.895636 11.503153 13.032967 14.990417 16.357368  1.553141 18.481447
	[78]  7.045387  8.081076 18.434614 17.275692 19.368941  3.098027  3.095955
	[85] 12.725635 17.716566 16.419147  2.190231 18.240423  7.308344 12.962402
	[92]  2.592744  5.913642  6.257219 12.704005  9.532116  2.970483 12.454670
	[99]  9.860412 17.268342
	> rt(5, df=10)
	[1]  1.5456106 -0.9824119  1.9551162  0.7600165 -0.7377210
	\end{minted}
	Notas:
	\begin{itemize}
		\item Por defeito, a função \mintinline{r}{rnorm} já devolve elementos tirados aleatoriamente de uma função normal padrão, tal como se pode verificar com a instrução \mintinline{r}{args(rnorm)};
		\item Estas funções devolvem sequências aleatórias e, como tal, é esperado que a sequência obtida em diferentes computadores, e/ou em diferentes instantes no tempo, seja diferente;
		\item Quando se avalia uma expressão na consola do \textbf{R}, a linha com o respectivo resultado, no caso de se tratar de um vector, começa com \mintinline{r}{[1]}. Esse número informa que se estão a visualizar os elementos pertencentes ao vector resultante, começando no primeiro. Noutro exemplo, no retorno da instrução acima \mintinline{r}{runif(100, 1, 20)}, pode-se facilmente perceber qual é o elemento na 74ª posição, bastando para tal localizar a linha que começa com \mintinline{r}{[71]} e contar 4 elementos para a direita, encontrando, assim, o valor \mintinline{r}{14.990417}.
	\end{itemize}
	
\end{solution}

\end{questions}

\section{Vectores -- Parte 2}
As sequências são uma forma útil de aceder, em simultâneo, a vários elementos de um vector. Tire proveito da indexação \textbf{com} vectores para responder às seguintes questões.
\begin{questions}
\question Obtenha as notas dos primeiros 3 alunos.
\begin{solution}
	\begin{minted}{R}
> notas[1:3]
[1] 10 12  8
	\end{minted}
\end{solution}

\question Liste os nomes de todos os alunos excepto dos últimos 3.
\begin{solution}
	\begin{minted}{R}
> nomes[-(5:7)]
[1] "André"    "Carolina" "João"     "Mariana" 
> nomes[-((length(nomes)-2):length(nomes))]
[1] "André"    "Carolina" "João"     "Mariana" 
	\end{minted}
	As duas soluções dão o mesmo resultado, mas a segunda tem a vantagem de funcionar independentemente do número de elementos no vector \mintinline{R}{notas}, enquanto que a primeira solução só funciona quando este vector tem 7 elementos.
\end{solution}

\question Quais são os alunos que tiveram positiva?
\begin{solution}
	\begin{minted}{R}
> nomes[notas >= 10]
[1] "André"    "Carolina" "Mariana"  "Pedro"    "Vânia"   
	\end{minted}
\end{solution}

\question Que notas houve entre 10 e 13?
\begin{solution}
	\begin{minted}{R}
> notas[notas >= 10 & notas <= 13]
[1] 10 12 13
	\end{minted}
\end{solution}

\question Descubra se há alunos muito diferentes, e quais os seus nomes, verificando se houve notas menores que 8 e maiores que 15.
\begin{solution}
	\begin{minted}{R}
> notas[notas < 8 | notas > 15]
[1] 16  7
> nomes[notas < 8 | notas > 15]
[1] "Pedro" "Zé"   
	\end{minted}
\end{solution}

\question Devolva as notas de 5 alunos escolhidos aleatoriamente.
\begin{solution}
	\begin{minted}{R}
> notas[runif(5, 1, length(notas))]
[1] 13 12 16 16 16
	\end{minted}
	A função \mintinline{R}{runif} devolve números reais, no entanto, durante a indexação de um vector, o \textbf{R} aceita e converte automaticamente números reais em números inteiros. Para a conversão ser explícita, pode-se usar a função \mintinline{R}{as.integer}:
	\begin{minted}{R}
	notas[as.integer(runif(5, 1, length(notas)))]
	\end{minted}
	Caso também se pretenda obter os nomes dos alunos, é necessário guardar os índices numa variável auxiliar, caso contrário, por serem escolhidos aleatoriamente, os nomes e as notas não corresponderão aos dados na \autoref{tab: pauta}:
	\begin{minted}{R}
> (indices.amostra = as.integer(runif(5, 1, length(notas))))
[1] 3 1 2 6 1
> notas[indices.amostra]
[1]  8 10 12 13 10
> nomes[indices.amostra]
[1] "João"     "André"    "Carolina" "Vânia"    "André"   
	\end{minted}
\end{solution}

\end{questions}

\section{Matrizes}
Uma matriz, em \textbf{R}, é a extensão de um vector para duas dimensões.

\begin{questions}
\question Crie um novo vector, \mintinline{r}{idades}, com as idades dos alunos.
\begin{solution}
	Pode optar por criar o vector idades com um conjunto de quaisquer valores. Alternativamente, pode dar uso às funções que criam sequências aleatórias para criar um conjunto de dados fictício. Por exemplo, admitindo que a idade dos alunos segue uma distribuição normal, com média 20 e desvio padrão 2:
	\begin{minted}{R}
> idades = as.integer(rnorm(n=length(nomes), mean=20, sd=2))
> idades
[1] 22 20 18 17 20 19 19
	\end{minted}
\end{solution}

\question Construa uma matriz, \mintinline{r}{pauta}, a partir dos vectores \mintinline{r}{notas} e \mintinline{r}{idades}.
\begin{solution}
	\begin{minted}{R}
> pauta = cbind(notas, idades)
> pauta
notas idades
[1,]    10     22
[2,]    12     20
[3,]     8     18
[4,]    15     17
[5,]    16     20
[6,]    13     19
[7,]     7     19
	\end{minted}
\end{solution}

\question Use o vector \mintinline{r}{nomes} para atribuir etiquetas às linhas da matriz \mintinline{r}{pauta}.
\begin{solution}
	\begin{minted}{R}
> rownames(pauta) = nomes
> pauta
notas idades
André       10     22
Carolina    12     20
João         8     18
Mariana     15     17
Pedro       16     20
Vânia       13     19
Zé           7     19
	\end{minted}
\end{solution}

\question Troque as linhas pelas colunas dessa matriz.
\begin{solution}
	\begin{minted}{R}
> t(pauta)
André Carolina João Mariana Pedro Vânia Zé
notas     10       12    8      15    16    13  7
idades    22       20   18      17    20    19 19
	\end{minted}
\end{solution}

\end{questions}
\end{document}