% !TeX spellcheck = pt_PT
%% ************************************************
%% Universidade Nova de Lisboa
%% NOVA Information Management School
%% Computação em Estatística e Gestão de Informação
%% Júlio Caineta, 2016
%% **********************************************
\documentclass{exam}
\usepackage[utf8]{inputenc}
\usepackage[portuguese]{babel}
\usepackage[hidelinks,pdfusetitle]{hyperref}
\author{Júlio Caineta}
\title{CEGI 2015/2016 - Exercícios 3}
\usepackage{lastpage}
\usepackage{minted}
\usepackage{paralist}
\usepackage{upquote}
\usepackage{amsmath}
%\usepackage{color}
\cfoot{Página \thepage\ de \pageref{LastPage}}
\renewcommand{\solutiontitle}{\noindent\textbf{Solução:}\par\noindent}
\newminted{r}{autogobble}
\newmintinline{r}{}
 
\printanswers
%\shadedsolutions

\begin{document}
 
\begin{center}
\textsc {\small NOVA IMS -- Universidade Nova de Lisboa} \\
\textsc {Computação em Estatística e Gestão de Informação -- 2º Semestre 2015/16}
\vspace{5mm} \\
{\large Exercícios 3}
\end{center}
 
\vspace{5mm}

\section*{Funções}

Em \textbf{R}, as funções são também objectos e podem ser manipuladas como tal. Têm três principais propriedades:

\begin{description}
	\item[argumentos] Conjunto dos parâmetros que a função vai receber, em que alguns podem ter um valor por defeito e ser opcionais. Uma forma rápida de consultar os argumentos de uma função é através da função \rinline{args}.
	\item[corpo]  Ao conjunto de expressões avaliado na execução da função chama-se corpo da função. É possível visualizar o corpo de uma função através da função \rinline{body}.
	\item[ambiente] Uma função é definida num determinado ambiente (\textit{environment}) activo, o que irá determinar os objectos visíveis e acessíveis pela função, também conhecido como alcance (\textit{scope}).
\end{description}
 
\begin{questions}
\question  Considere o seguinte \rinline{data.frame} que contém uma pauta com as notas dos alunos de CEGI:

\begin{rcode}
	> dput(pauta)
	structure(list(`Teste SAS` = c(14L, 14L, 13L, 15L), `Teste R` = c(13L,
	19L, 10L, 12L), `Trabalho 1` = c(18L, 6L, 16L, 9L), `Trabalho 2` = c(11L,
	14L, 12L, 9L)), .Names = c("Teste SAS", "Teste R", "Trabalho 1",
	"Trabalho 2"), row.names = c("Rui", "Manuel", "Mariana", "Carolina"
	), class = "data.frame")
\end{rcode}

A seguinte função, usa uma estrutura iterativa para encontrar o nome dos alunos que tiveram positiva ($\geq 10$) em todas as componentes.

\begin{rcode}
tudo_positivo = function (notas) {
	aluno = 1
	nomes = c()
	while (aluno <= nrow(notas)) {
		if (all(notas[aluno, ] >= 10)) {
			nomes = c(nomes, rownames(notas)[aluno])
		}
		aluno = aluno + 1
	}
	return (nomes)
}

> tudo_positivo(pauta)
[1] "Rui"     "Mariana"
\end{rcode}

O \textbf{R} é uma linguagem versátil e compatível com o paradigma de programação funcional, sendo possível resolver este exercício de uma forma mais curta e eficiente (além de elegante). Reescreva a função \rinline{tudo_positivo} sem usar uma estrutura iterativa.

{\small \textbf{Nota:} A função \rinline{dput()} é útil para recriar objectos definidos em R. Neste exemplo, o \rinline{data.frame pauta} estava guardado numa sessão em R, e com esta função é possível guardar uma expressão que permite recriar o mesmo objecto. Assim, para ter esta pauta na sua sessão, basta copiar o resultado da chamada da função \rinline{dput()}}.

\begin{solution}
\begin{rcode}
tudo_positivo = function(notas) {
	rownames(notas)[apply(notas, 1, function(x) all(x >= 10))]
}
\end{rcode}
\end{solution}

\question A função \rinline{mais_faceis} mostra as componentes em que mais alunos (por defeito, pelo menos 3 alunos) tiveram positiva. Esta função utiliza a coerção de valores lógicos para valores numéricos.

\begin{rcode}
	mais_faceis = function(notas, lim=3) {
		componentes = c()
		for (componente in 1:ncol(notas)) {
			if (sum(notas[, componente] >= 10) >= lim) {
				componentes = c(componentes, (colnames(notas)[componente]))
			}
		}
		return (componentes)
	}
	
	> mais_faceis(pauta)
	[1] "Teste SAS"  "Teste R"    "Trabalho 2"
\end{rcode}

Reescreva esta função usando o paradigma da programação funcional (sem estruturas iterativas).

\begin{solution}
	\begin{rcode}
		mais_faceis = function(notas, lim=3) {
			colnames(notas)[apply(notas, 2, function(x) sum(x >= 10) >= lim)]
		}
	\end{rcode}
\end{solution}

\question O vector \rinline{rivers} contém o comprimento, em milhas, dos 141 principais rios norte americanos.

Escreva a função \rinline{perc}, que recebe dois argumentos: um vector numérico, \rinline{v} e um número, \rinline{x}. A função devolve o percentil de \rinline{v} correspondente ao valor \rinline{x}.

Por exemplo, considerando o vector \rinline{rivers}:

\begin{rcode}
	> perc(rivers, 680)
	[1] 75.1773
	> perc(rivers, 425)
	[1] 50.35461
	> perc(rivers, 10)
	[1] 0
	> perc(rivers, 10000)
	[1] 100
\end{rcode}

Estes valores correspondem aos obtidos com a função \rinline{quantile}.

\begin{rcode}
	> quantile(rivers)
	0%    25%    50%    75%    100% 
	135   310    425    680    3710 
\end{rcode}

{\footnotesize \textit{Adaptado do Exame de Época Especial de 2014/2015.}}

\begin{solution}
	\begin{rcode}
		perc = function (v, x) {
			return (length(subset(v, v <= x)) / length(v) * 100)
		}
	\end{rcode}
\end{solution}

\question Atente ao seguinte bloco de código:
\begin{rcode}
	x = 2
	z = 56.3
	f = function(x) {
		a = 34
		y = x / 4 * a * z
		y
	}
	f(21)
	a
\end{rcode}
Qual o valor das variáveis \rinline{a}, \rinline{x} e \rinline{z} dentro do ambiente da função? O que aconteceu ao valor da variável \rinline{a} após a execução da função?

\begin{solution}
	\begin{rcode}
		f = function(x) {
			a = 34
			y = x / 4 * a * z
			cat("Variável x: ", x, "\nVariável a: ", a, "\nVariável z: ", z, "\n")
			y
		}
		
		> x = 2
		> z = 56.3
		> f(21)
		Variável x:  21 
		Variável a:  34 
		Variável z:  56.3 
		[1] 10049.55
	\end{rcode}
	Os valores visualizados correspondem aos valores que aquelas variáveis têm \textbf{dentro} da função:
	\begin{itemize}
		\item A variável \rinline{x} toma o valor passado como argumento da função. Não é o mesmo objecto que existe fora da função, embora tenha o mesmo nome (existem em ambientes diferentes).
		\item A variável \rinline{a} só existe dentro da função e tem o valor que recebe dentro desta.
		\item A variável \rinline{z} existe fora da função, mas quando acedida dentro da função, como não existe nenhum outro objecto com o mesmo nome, é procurado um objecto com esse nome no ambiente imediatamente acima do ambiente da função \rinline{f}.
	\end{itemize}
\end{solution}

\question Defina uma função para calcular o factorial (\autoref{eq:fact}) de um dado número natural.

\begin{equation}
\text{fact}(n) =
	\begin{cases}
		1 & \text{se } n = 0 \\
		n \times \text{fact}(n - 1) & \text{se } n > 0
	\end{cases}
	\label{eq:fact}
\end{equation}

\vspace{5mm}
\begin{quote}
	Recursion in computer science is a method where the solution to a problem depends on solutions to smaller instances of the same problem (as opposed to iteration).
\end{quote}
\vspace{5mm}
\rightline{{\rm --- Graham et al., 1990}}

Agora defina a mesma função mas utilizando uma definição recursiva.

\begin{solution}
	\begin{rcode}
		# definição iterativa
		factIter = function(n) {
			f = 1
			for (i in 2:n) f = f * i
			f
		}
		# definição recursiva
		fact = function(n) {
			if (n <= 1) 1
			else n * fact(n - 1)
		}
	\end{rcode}
\end{solution}

\question Dado um vector \rinline{x} com números inteiros, construa um novo vector com a soma dos números iguais que existem em \rinline{x}.

Por exemplo, no seguinte vector 

\begin{rcode}
	x = (1, 2, 5, 4, 3, 1, 1, 4, 2, 6, 7, 2, 4, 1, 5)
\end{rcode}

a soma dos números 1 dá 4 (há quatro 1), a soma dos números 2 dá 6 (há três 2), a soma dos números 3 dá 3 (há um 3), etc., tal que o resultado esperado é o seguinte:

\begin{rcode}
	[1]  4  6  3 12 10  6  7
\end{rcode}

\begin{solution}
	\begin{rcode}
		> sapply(sort(unique(x)), function(i) sum(x[x == i]))
		[1]  4  6  3 12 10  6  7
	\end{rcode}
	\end{solution}

\end{questions}

\end{document}